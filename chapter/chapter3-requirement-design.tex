% 第3章 基于大语言模型的 AI Agent 服务后端系统需求分析与设计
\chapter{AI Agent 服务后端系统的需求分析与设计}
\label{chap:reqdesign}

\section{系统概述}
\subsection{业务场景与目标}
本节可结合企业自研 AI 生活搜索助手,介绍系统所处的业务场景、目标用户及核心使用流程。

\subsection{功能定位与角色划分}
本节可说明 Agent 服务后端在整体产品中的位置,以及与前端、搜索引擎、审查服务等模块的关系。

\section{需求分析}
\subsection{功能需求分析}
本节可从会话管理、长时记忆、ReAct 流程编排、工具调用、安全审查、RAG 检索等方面分条列出系统的主要功能需求。

\subsection{非功能需求分析}
本节可从可用性、性能、扩展性、可靠性、安全性、可运维性与成本控制等维度分析系统的非功能需求。

\section{系统总体架构设计}
\subsection{总体架构设计}
本节可给出系统的整体架构图,说明各模块以及与外部系统之间的交互关系。

\subsection{模块划分与职责}
本节可对长时记忆子系统、Agent 编排引擎、工具管理与调度模块、RAG 子系统、内容安全模块、观测与日志模块等进行概要说明。

\section{核心业务流程设计}
\subsection{单轮/多轮对话处理流程}
本节可从用户请求进入系统开始,详细描述请求在网关、Agent 引擎、工具层与模型调用之间的传递路径。

\subsection{ReAct 推理与工具并行调用流程}
本节可重点说明 ReAct 循环中 Prompt 生成、模型输出解析、并行工具调用、结果合并与错误恢复等关键设计。

\section{数据与存储设计}
\subsection{逻辑数据模型}
本节可描述用户、会话、记忆条目、工具调用记录、Token 账单等核心实体及其关系。

\subsection{数据库与向量库设计}
本节可给出 MySQL 与 Milvus 中的关键表/集合与索引设计,并说明其与业务需求之间的对应关系。


