% 第4章 AI Agent 服务后端系统的实现
\chapter{AI Agent 服务后端系统的实现}
\label{chap:impl}

\section{总体实现框架}
\subsection{技术选型与工程环境}
本节可介绍编程语言、框架、依赖组件以及部署环境等总体实现选择。

\subsection{系统分层与接口设计}
本节可说明表示层、服务层、领域层与基础设施层的划分,以及各层之间的接口约定。

\section{长时记忆子系统实现}
\subsection{记忆层次与写入策略}
本节可介绍短期记忆、工作记忆与长期记忆的划分,以及记忆写入、更新与归档策略。

\subsection{记忆检索与召回策略}
本节可说明如何基于向量检索与关键字检索进行记忆召回,以及与会话上下文融合的方式。

\section{ReAct 调度与工具调用实现}
\subsection{Prompt 编排与思维链记录}
本节可介绍 Prompt 模板设计、思维链记录格式以及如何支持执行过程回放。

\subsection{工具注册、权限控制与并行调度}
本节可详细说明工具元数据管理、权限与配额控制机制,以及基于异步编程模型实现工具并行调用的过程。

\section{Token 成本治理与上下文压缩实现}
\subsection{Token 统计与账单记录}
本节可介绍请求级与会话级的 Token 统计、阈值告警与成本看板等实现。

\subsection{上下文裁剪与摘要生成}
本节可说明在保证效果前提下,通过窗口裁剪、语义聚合与自动摘要等方式减少上下文长度的实现方案。

\section{RAG 融合与内容安全审查实现}
\subsection{RAG 流程与企业搜索系统集成}
本节可介绍与企业自研搜索系统的接口设计、召回与排序策略,以及与模型推理的融合方式。

\subsection{同步与异步内容安全审查链路}
本节可说明多维度审查规则配置、同步/异步审查组合策略以及策略热更新机制的实现。


