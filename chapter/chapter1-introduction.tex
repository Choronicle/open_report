% 第1章 绪论
\chapter{绪论}
\label{chap:intro}

% 1.1 项目背景
\section{项目背景}
随着大语言模型(Large Language Models, LLM)能力的快速提升,AI Agent 已从简单的对话机器人演变为能够自主规划、调用外部工具、执行多步骤复杂任务的智能实体,在智能客服、自动化工作流、代码生成、科研辅助等领域展现出巨大的应用价值\,[1]。近年来,国内外科技巨头纷纷推出具备 Agent 能力的商业产品,例如 OpenAI 的 GPTs、Microsoft Copilot、Google Gemini Live,以及国内的豆包、文心一言等。这些产品在实际运行中普遍遇到从原型到稳定服务的一系列工程问题,促使业界更加关注 Agent 后端系统的生产级实现\,[2]。

本课题源于企业内部自研的一款独立 AI 生活搜索助手,该助手基于大语言模型,结合 ReAct 推理框架与企业自有搜索系统,为用户提供带工具调用和知识检索的智能问答服务。在开发和迭代过程中,后端系统逐步暴露出若干 Agent 特有的工程难点,主要包括:

(1)长时记忆管理困难:现有框架依赖模型上下文窗口,超过限制后历史信息即永久丢失,用户需反复描述背景,导致体验极差。

(2)ReAct 执行链路不透明与响应慢:推理-行动-观察循环中可能因工具失败、模型幻觉或搜索返回空结果而陷入死循环,中间过程黑盒化;同时工具调用串行等待,整体响应时间较长。

(3)工具调用安全隐患与 Token 成本失控:缺乏统一的安全隔离和权限控制,复杂任务常因多次重试、长上下文或冗余检索导致单次调用消耗数万 Token,成本难以预测和控制。

(4)RAG 融合困难与内容安全审查实时性不足:需与企业自有搜索系统深度对接,涉及召回融合、缓存策略等问题;审查策略频繁调整,但现有方案难以实现热更新与异步处理。

现有开源框架如 LangChain\,[12]、LangGraph\,[13]、Microsoft AutoGen\,[10] 等虽显著降低了 Agent 开发门槛,但在生产环境下的稳定性、异步支持、可观测性以及上述痛点方面仍存在明显不足\,[8]。基于上述企业实际项目背景,在“降本增效、AI 驱动业务增长”的大方向下,如何基于 Spring AI Alibaba 构建一套生产可用的、企业级 AI Agent 服务后端系统,实现长时记忆分层管理、ReAct 异步并行执行、工具安全调度、Token 成本治理、动态 RAG 配额控制以及内容安全审查,已成为当前大模型中台团队迫切需要解决的核心问题。

因此,本文从生产级 AI Agent 服务后端系统的角度出发,设计并实现了一个高可用、可扩展、可观测的 AI Agent 服务框架。本文设计的系统能帮助开发人员自动管理长时记忆、优化 ReAct 执行流程、确保工具调用安全合规,并提供 Token 消耗实时统计与压缩机制。同时,为了方便管理人员了解系统运行状态,本文在后台对每日产生的调用日志、性能指标、异常事件进行分类汇总与可视化展示,及时向管理人员推送告警并生成报告,为业务方的 Agent 能力优化与资源规划提供数据支撑,最终大幅提升企业 AI Agent 的落地效率与生产稳定性。

% 1.2 国内外研究现状
\section{国内外研究现状}
\subsection{大语言模型发展现状}
近年来,OpenAI、Anthropic、Google 与国内多家厂商陆续发布了多代大语言模型,在长上下文理解、工具调用、结构化输出等方面取得了显著进展。这些模型为构建具备自主推理与行动能力的 AI Agent 提供了基础。

\subsection{AI Agent 框架研究现状}
为弥补大语言模型行动能力的不足,学术界和工业界提出了多种 Agent 框架,例如 ReAct、AutoGen、MemGPT 以及面向工程实践的 LangChain、LangGraph 等。这些框架在工具管理、多 Agent 协作、记忆增强等方面进行了大量探索,但多数工作仍停留在研究原型或小规模应用阶段。

\subsection{工程化 Agent 后端系统研究}
近年来,围绕大规模多 Agent 系统、记忆增强 Agent 与可观测性等问题也出现了一批研究工作。然而,相比于模型与算法层面的研究,针对生产级 Agent 服务后端架构、容错机制、性能优化与成本治理等工程化课题的系统性研究仍然相对匮乏。

% 1.3 研究内容与目标
\section{研究内容与目标}
本课题以企业内部自研的 AI 生活搜索助手为背景,聚焦其基于大语言模型的 Agent 服务后端系统。围绕长时记忆管理、ReAct 推理流程支持、工具安全调度、Token 成本控制、RAG 融合及内容安全审查等关键工程问题,本文拟开展以下几个方面的研究与实现:
\begin{itemize}
  \item 设计并实现支持短期记忆、工作记忆与长期记忆分层管理的长时记忆子系统;
  \item 构建支持工具并行调用与执行过程可回放的 ReAct 调度引擎;
  \item 建立统一的工具注册与权限控制机制,实现安全可控的函数调用;
  \item 构建 Token 使用实时统计与上下文压缩机制,实现模型调用成本治理;
  \item 实现与企业自研搜索系统深度融合的 RAG 流程以及多维度内容安全审查链路。
\end{itemize}

% 1.4 论文结构安排
\section{论文结构安排}
本文余下部分的组织结构如下:
\begin{itemize}
  \item 第~\ref{chap:tech} 章对大语言模型、AI Agent 框架及相关支撑技术进行综述;
  \item 第~\ref{chap:reqdesign} 章给出基于实际业务场景的需求分析,并在此基础上完成系统总体设计;
  \item 第~\ref{chap:impl} 章详细介绍 Agent 服务后端系统的关键模块实现;
  \item 第~\ref{chap:experiment} 章对系统的功能、性能与成本治理效果进行测试与分析;
  \item 第~\ref{chap:conclusion} 章对全文工作进行总结,并展望未来的研究方向。
\end{itemize}
