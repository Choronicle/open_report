% 第1章 绪论
\chapter{绪论}
\label{chap:intro}

% 1.1 研究背景
\section{研究背景}
随着大语言模型能力的快速提升,AI Agent 已从简单的对话机器人演变为能够自主规划、调用外部工具、执行多步骤复杂任务的智能实体,在智能客服、自动化工作流、代码生成、科研辅助等领域展现出巨大的应用价值。

在实际企业级部署中,Agent 后端系统需要同时面对高并发请求、首 Token 耗时优化、长时序状态管理、工具调用安全、跨会话记忆持久化、调用成本控制以及全链路可观测性等诸多工程挑战。现有工程实践往往依赖手工脚本或对开源框架的简单封装,难以满足日调用量十万级甚至百万级别的生产要求。这些问题构成本课题研究的直接工程背景。

% 1.2 国内外研究现状
\section{国内外研究现状}
\subsection{大语言模型发展现状}
近年来,OpenAI、Anthropic、Google 与国内多家厂商陆续发布了多代大语言模型,在长上下文理解、工具调用、结构化输出等方面取得了显著进展。这些模型为构建具备自主推理与行动能力的 AI Agent 提供了基础。

\subsection{AI Agent 框架研究现状}
为弥补大语言模型行动能力的不足,学术界和工业界提出了多种 Agent 框架,例如 ReAct、AutoGen、MemGPT 以及面向工程实践的 LangChain、LangGraph 等。这些框架在工具管理、多 Agent 协作、记忆增强等方面进行了大量探索,但多数工作仍停留在研究原型或小规模应用阶段。

\subsection{工程化 Agent 后端系统研究}
近年来,围绕大规模多 Agent 系统、记忆增强 Agent 与可观测性等问题也出现了一批研究工作。然而,相比于模型与算法层面的研究,针对生产级 Agent 服务后端架构、容错机制、性能优化与成本治理等工程化课题的系统性研究仍然相对匮乏。

% 1.3 研究内容与目标
\section{研究内容与目标}
本课题以企业内部自研的 AI 生活搜索助手为背景,聚焦其基于大语言模型的 Agent 服务后端系统。围绕长时记忆管理、ReAct 推理流程支持、工具安全调度、Token 成本控制、RAG 融合及内容安全审查等关键工程问题,本文拟开展以下几个方面的研究与实现:
\begin{itemize}
  \item 设计并实现支持短期记忆、工作记忆与长期记忆分层管理的长时记忆子系统;
  \item 构建支持工具并行调用与执行过程可回放的 ReAct 调度引擎;
  \item 建立统一的工具注册与权限控制机制,实现安全可控的函数调用;
  \item 构建 Token 使用实时统计与上下文压缩机制,实现模型调用成本治理;
  \item 实现与企业自研搜索系统深度融合的 RAG 流程以及多维度内容安全审查链路。
\end{itemize}

% 1.4 论文结构安排
\section{论文结构安排}
本文余下部分的组织结构如下:
\begin{itemize}
  \item 第~\ref{chap:tech} 章对大语言模型、AI Agent 框架及相关支撑技术进行综述;
  \item 第~\ref{chap:reqdesign} 章给出基于实际业务场景的需求分析,并在此基础上完成系统总体设计;
  \item 第~\ref{chap:impl} 章详细介绍 Agent 服务后端系统的关键模块实现;
  \item 第~\ref{chap:experiment} 章对系统的功能、性能与成本治理效果进行测试与分析;
  \item 第~\ref{chap:conclusion} 章对全文工作进行总结,并展望未来的研究方向。
\end{itemize}
