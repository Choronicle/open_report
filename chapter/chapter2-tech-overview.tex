% 第2章 相关技术综述
\chapter{相关技术与平台综述}
\label{chap:tech}

\section{大语言模型技术概述}
\subsection{大语言模型发展历程}
本节可回顾从 GPT 系列到最新多模态模型的发展脉络,总结参数规模、训练数据和能力演进。

\subsection{长上下文与工具调用能力}
本节可介绍长上下文建模、函数调用(Tool Calling)、结构化输出等与 Agent 紧密相关的模型能力。

\section{AI Agent 框架与典型系统}
\subsection{ReAct 推理框架}
本节可系统介绍 ReAct 框架的“推理-行动-观察”闭环流程,以及其在工具调用和任务分解方面的优势与局限。

\subsection{典型 Agent 框架}
本节可分别概述 LangChain、LangGraph、AutoGen、MemGPT 等代表性框架的核心思想、功能特性及适用场景。

\section{支撑技术与基础设施}
\subsection{数据与向量存储}
本节可介绍 MySQL、Milvus 等在本系统中的角色,说明结构化数据与向量检索在长时记忆、RAG 流程中的作用。

\subsection{缓存与消息中间件}
本节可介绍 Redis 等组件在会话状态缓存、限流与异步任务处理中的应用。

\subsection{服务框架与工程实践}
本节可介绍 Spring AI 与 Spring AI Alibaba 等框架的基本概念,说明其在模型调用抽象、配置管理与监控集成方面提供的支持。


