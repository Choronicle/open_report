% 第2章 相关技术综述
\chapter{相关技术与平台综述}
\label{chap:tech}

\section{大语言模型技术概述}
\subsection{大语言模型发展历程}
本节可回顾从 GPT 系列到最新多模态模型的发展脉络,总结参数规模、训练数据和能力演进。

\subsection{长上下文与工具调用能力}
本节可介绍长上下文建模、函数调用(Tool Calling)、结构化输出等与 Agent 紧密相关的模型能力。

\section{AI Agent 框架与典型系统}
\subsection{ReAct 推理框架}
本节可系统介绍 ReAct 框架的“推理-行动-观察”闭环流程,以及其在工具调用和任务分解方面的优势与局限。

\subsection{典型 Agent 框架}
本节可分别概述 LangChain、LangGraph、AutoGen、MemGPT 等代表性框架的核心思想、功能特性及适用场景。

\section{数据存储技术}
\subsection{关系型数据库MySQL}
本节可介绍 MySQL 在本系统中的角色,说明结构化数据在长时记忆、RAG 流程中的作用。

\subsection{向量数据库Milvus}
本节可介绍 Milvus 在本系统中的角色,说明向量检索在长时记忆、RAG 流程中的作用。

\subsection{配置中心Apollo}
本节可介绍 Apollo 在本系统中的角色,说明配置管理在系统中的作用。

\subsection{分布式缓存Redis}
本节可介绍 Redis 在本系统中的角色,说明缓存机制在系统中的作用。

\section{服务框架与工程实践}
\subsection{Spring AI}
本节可介绍 Spring AI 在本系统中的角色,说明其在模型调用抽象、配置管理与监控集成方面提供的支持。

\subsection{Spring AI Alibaba}
本节可介绍 Spring AI Alibaba 在本系统中的角色,说明其在模型调用抽象、配置管理与监控集成方面提供的支持。

