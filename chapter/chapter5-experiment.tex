% 第5章 系统测试与评估
\chapter{系统测试与评估}
\label{chap:experiment}

\section{测试环境与指标体系}
\subsection{测试环境配置}
本节可介绍硬件环境、软件版本、部署拓扑,以及测试数据集与流量构造方式。

\subsection{评估指标设计}
本节可从功能正确性、响应时延、吞吐量、稳定性、Token 成本等维度设计量化指标。

\section{功能测试}
\subsection{核心功能验证}
本节可对会话管理、长时记忆、工具调用、RAG 检索、内容安全等核心功能进行用例验证。

\subsection{异常场景与容错测试}
本节可设计工具超时、模型调用失败、外部依赖异常等场景,验证系统的容错与恢复能力。

\section{性能与可扩展性测试}
\subsection{并发性能测试}
本节可通过压力测试工具模拟不同并发级别下的请求,分析首 Token 时延与整体吞吐表现。

\subsection{扩展性与资源利用率分析}
本节可在不同节点规模与配置下,比较系统的扩展效率与资源利用率。

\section{成本治理效果分析}
\subsection{Token 成本变化对比}
本节可在开启与关闭上下文压缩、记忆裁剪等策略的条件下,对比单次与整体 Token 消耗。

\subsection{综合评估与讨论}
本节可综合分析性能、成本与用户体验之间的权衡,总结系统优化带来的收益与不足。


