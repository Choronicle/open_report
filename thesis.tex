% !Mode:: "TeX:UTF-8"
%# -*- coding:utf-8 -*-

%% 南京大学学位论文的示例文档
%% 作者:njuhan: https://github.com/njuHan
%% 源模版repo: https://github.com/njuHan/njuthesis-nju-thesis-template

\documentclass[winfonts,master,twoside,AutoFakeBold= {2}]{njuthesis}
%% njuthesis 文档类的可选参数有:
%%   winfonts, linuxfonts, macfonts, adobefonts winfonts 选项使得文档使用Windows 系统提供的字体;linuxfonts 选项使得文档使用Linux 系统提供的字体;macfonts 选项使得文档使用Mac 系统提供的字体;adobefonts 选项使得文档使用Adobe提供的OTF中文字体(需自行下载安转)
%%   phd/master/bachelor 选择博士/硕士/学士论文
%%   twoside 或 oneside 指定排版的文档为双面打印或单面打印格式(twoside会使得chapter 章节从奇数页开始,即纸张的正面开始,因此会出现一些空白的页面)
%%   nobackinfo 取消封二页导师签名信息。注意,按照南大的规定,是需要签名页的。
%%   AutoFakeBold 设置CJK字体加粗,参数“2”用于指定加粗程度(空格勿删,删除会引发编译错误)


%%%%%%%%%%%%%%%%%%%%%%%%%%%%%%%%%%%%%%%%%%%%%%%%%%%%%%%%%%%%%%%%%%%%%%%%%%%%%%%
% set up labelformat and labelsep for subfigure 详见: http://www.latexstudio.net/archives/8652.html
\captionsetup[subfigure]{labelformat=simple, labelsep=space}

%%%%%%%%%%%%%%%%%%%%%%%%%%%%%%%%%%%%%%%%%%%%%%%%%%%%%%%%%%%%%%%%%%%%%%%%%%%%%%%
% 设置《国家图书馆封面》的内容,仅博士论文才需要填写

% 设置论文按照《中国图书资料分类法》的分类编号
\classification{0175.2}
% 设置论文按照《国际十进分类法UDC》的分类编号
% 该编号可在下述网址查询:https://udcsummary.info/php/index.php?lang=chi
\udc{004.72}
% 国家图书馆封面上的论文标题第一行,不可换行。此属性可选,默认值为通过\title设置的标题。
\nlctitlea{论文标题第一行}
% 国家图书馆封面上的论文标题第二行,不可换行。此属性可选,默认值为空白。
\nlctitleb{论文标题第二行}
% 国家图书馆封面上的论文标题第三行,不可换行。此属性可选,默认值为空白。
\nlctitlec{}
% 导师的单位名称及地址
\supervisorinfo{南京大学计算机科学与技术系~~南京市汉口路22号~~210093}
% 答辩委员会主席
\chairman{张三丰~~教授}
% 第一位评阅人
\reviewera{阳顶天~~教授}
% 第二位评阅人
\reviewerb{张无忌~~副教授}
% 第三位评阅人
\reviewerc{黄裳~~教授}
% 第四位评阅人
\reviewerd{郭靖~~研究员}


%%%%%%%%%%%%%%%%%%%%%%%%%%%%%%%%%%%%%%%%%%%%%%%%%%%%%%%%%%%%%%%%%%%%%%%%%%%%%%%
% 设置论文的中文封面

% 单行论文标题,不可换行


% 如果论文标题过长,可以分两行,第一行用\titlea{}定义,第二行用\titleb{}定义,
% 使用以下3行:
%\title{} %用于覆盖单行标题内容为空
%\titlea{长标题第一行}  %第一行标题写这里
%\titleb{长标题第二行用于长标题换行} %第二行标题写这里
% 注意: \title 不能都注释,它用于控制标题选择双行还是单行。\title{}如果内容为空,则编译\titlea{},titleb{}双行标题,否则编译单行标题
\title{}
\titlea{基于大语言模型的 AI Agent 服务}
\titleb{后端系统的设计与实现}

% 论文作者姓名
\author{王非}
% 论文作者联系电话
\telphone{xxxx}
% 论文作者电子邮件地址
\email{sample@smail.nju.edu.cn}
% 论文作者学生证号
\studentnum{xxxxxxxxxx}
% 论文作者入学年份(年级)
\grade{2024}
% 论文作者毕业年份(届), 出版授权书的学位年度
\graduateyear{2026}
% 导师姓名职称
\supervisor{潘敏学教授}
% 导师的联系电话
\supervisortelphone{}
% 论文作者的学科与专业方向
\major{软件工程}
% 论文作者的研究方向
\researchfield{软件工程}
% 论文作者所在院系的中文名称
\department{软件学院}
% 论文作者所在学校或机构的名称。此属性可选,默认值为``南京大学''。
\institute{南京大学}
% 论文的提交日期,需设置年、月、日。
\submitdate{xxxx年 xx 月 xx 日}
% 论文的答辩日期,需设置年、月、日。
\defenddate{xxxx年 xx 月 xx 日}
% 论文的定稿日期,需设置年、月、日。
% 此属性可选,若注释\date{},则默认值为最后一次编译时的日期,精确到日。
% \date{2019年5月20日}

%%%%%%%%%%%%%%%%%%%%%%%%%%%%%%%%%%%%%%%%%%%%%%%%%%%%%%%%%%%%%%%%%%%%%%%%%%%%%%%
% 设置论文的英文封面

% 论文的英文标题,不可换行
\englishtitle{The Design and Implementation of a Backend Service System for a Large Language Model-Based AI Agent Application}
% 论文作者姓名的拼音
\englishauthor{Fei Wang}
% 导师姓名职称的英文
\englishsupervisor{Professor Minxue Pan}
% 论文作者学科与专业的英文名
\englishmajor{Software Engineering}
% 论文作者所在院系的英文名称
\englishdepartment{Software Institute}
% 论文作者所在学校或机构的英文名称。此属性可选,默认值为``Nanjing University''。
\englishinstitute{Nanjing University}
% 论文完成日期的英文形式,它将出现在英文封面下方。需设置年、月、日。日期格式使用美国的日期
% 格式,即``Month day, year'',其中``Month''为月份的英文名全称,首字母大写;``day''为
% 该月中日期的阿拉伯数字表示;``year''为年份的四位阿拉伯数字表示。
% 此属性可选,若注释掉\englishdate{},则默认值为最后一次编译时的日期。
% \englishdate{May 20, 2019}

%%%%%%%%%%%%%%%%%%%%%%%%%%%%%%%%%%%%%%%%%%%%%%%%%%%%%%%%%%%%%%%%%%%%%%%%%%%%%%%
% 设置论文的中文摘要

% 设置中文摘要页面的论文标题及副标题的第一行。
% 此属性可选,其默认值为使用|\title|命令所设置的论文标题
\abstracttitlea{基于大语言模型的 AI Agent 服务}
% 设置中文摘要页面的论文标题及副标题的第二行。
% 此属性可选,其默认值为空白
\abstracttitleb{后端系统的设计与实现}

%%%%%%%%%%%%%%%%%%%%%%%%%%%%%%%%%%%%%%%%%%%%%%%%%%%%%%%%%%%%%%%%%%%%%%%%%%%%%%%
% 设置论文的英文摘要

% 设置英文摘要页面的论文标题及副标题的第一行。
% 此属性可选,其默认值为使用|\englishtitle|命令所设置的论文标题
\englishabstracttitlea{The Design and Implementation of a Backend Service System}
% 设置英文摘要页面的论文标题及副标题的第二行。
% 此属性可选,其默认值为空白
\englishabstracttitleb{for a Large Language Model-Based AI Agent Application}

%%%%%%%%%%%%%%%%%%%%%%%%%%%%%%%%%%%%%%%%%%%%%%%%%%%%%%%%%%%%%%%%%%%%%%%%%%%%%%
%% 盲审命令,空白字段设置请看 .cls文件 \newcommand*{\blind}
%% 此外,请按照盲审要求自行去掉个人简历、致谢等页面中的个人信息
%\blind

%%%%%%%%%%%%%%%%%%%%%%%%%%%%%%%%%%%%%%%%%%%%%%%%%%%%%%%%%%%%%%%%%%%%%%%%%%%%%%%
\begin{document}

%%%%%%%%%%%%%%%%%%%%%%%%%%%%%%%%%%%%%%%%%%%%%%%%%%%%%%%%%%%%%%%%%%%%%%%%%%%%%%%

% 制作国家图书馆封面(博士学位论文才需要)
%\makenlctitle
% 制作中文封面
\maketitle
% 制作英文封面
\makeenglishtitle


%%%%%%%%%%%%%%%%%%%%%%%%%%%%%%%%%%%%%%%%%%%%%%%%%%%%%%%%%%%%%%%%%%%%%%%%%%%%%%%
% 开始前言部分
\frontmatter

%%%%%%%%%%%%%%%%%%%%%%%%%%%%%%%%%%%%%%%%%%%%%%%%%%%%%%%%%%%%%%%%%%%%%%%%%%%%%%%
% 论文的中文摘要
\begin{abstract}
随着大语言模型能力的快速提升,AI Agent 已从简单的对话机器人演变为能够自主规划、调用外部工具、执行多步骤复杂任务的智能实体,在智能客服、自动化工作流、代码生成、科研辅助等领域展现出巨大的应用价值。然而,当前大多数Agent项目仍停留在研究原型或前端 Demo 阶段,缺乏稳定、可扩展、面向生产环境的后台服务支撑。在实际企业级部署中,Agent 后端系统需要同时面对高并发请求、首Token耗时优化、长时序状态管理、工具调用安全、跨会话记忆持久化、调用成本控制以及全链路可观测性等诸多工程挑战,传统的手工调优、简单脚本堆叠或直接复用开源框架的方式已难以满足日调用量十万级甚至百万级别的生产要求。近年来,虽然 LangChain、LangGraph、Microsoft AutoGen 等开源框架显著降低了 Agent 开发门槛,但针对生产级后端系统的架构设计、容错机制、性能优化与运维体系的研究仍然相对匮乏。研究如何设计并实现一套高可用、可扩展、易运维的 Agent 服务后端系统,从工程实践角度解决大语言模型驱动的智能体在真实生产环境中的落地难题,已成为提升企业智能化水平、加速 AIGC 技术产业化的重要途径。

%通过改变链路中子流的个数,分配不同的数据流量给不同的链路。

% 中文关键词。关键词之间用中文全角分号隔开,末尾无标点符号。
\keywords{大语言模型;智能体}
\end{abstract}

%%%%%%%%%%%%%%%%%%%%%%%%%%%%%%%%%%%%%%%%%%%%%%%%%%%%%%%%%%%%%%%%%%%%%%%%%%%%%%%
% 论文的英文摘要
\begin{englishabstract}
  With the rapid advancement of large language model capabilities, AI Agents have evolved from simple conversational bots into intelligent entities capable of autonomous planning, external tool invocation, and execution of multi-step complex tasks. They have demonstrated tremendous application value in areas such as intelligent customer service, automated workflows, code generation, and scientific research assistance [1]. However, most current Agent projects remain at the stage of research prototypes or frontend demos, lacking stable, scalable, and production-ready backend service support. In real-world enterprise deployments, Agent backend systems must simultaneously address numerous engineering challenges, including high-concurrency request handling, first-token latency optimization, long-term conversational state management, secure tool calling, cross-session memory persistence, cost control of model inference, and full-stack observability. Traditional approaches—such as manual tuning, ad-hoc scripting, or direct reuse of open-source frameworks—are increasingly inadequate for production environments with daily invocation volumes reaching hundreds of thousands or even millions.
  In recent years, although open-source frameworks such as LangChain [12], LangGraph [13], and Microsoft AutoGen [10] have significantly lowered the barrier to Agent development, research on production-grade backend system architecture, fault tolerance mechanisms, performance optimization, and operational maintainability remains relatively scarce. Investigating how to design and implement a highly available, scalable, and easily maintainable Agent service backend system—and thereby addressing, from an engineering practice perspective, the deployment challenges of large language model-driven intelligent agents in real production environments—has become a critical path toward enhancing enterprise intelligence and accelerating the industrialization of AIGC technologies.

%Rate adaptation can be implemented by adjusting the number of subflows on each path.

% 英文关键词。关键词之间用英文半角逗号隔开,末尾无符号。
\englishkeywords{Large Language Model, Agent}
\end{englishabstract}

%%%%%%%%%%%%%%%%%%%%%%%%%%%%%%%%%%%%%%%%%%%%%%%%%%%%%%%%%%%%%%%%%%%%%%%%%%%%%%%
% 论文的前言,应放在目录之前,中英文摘要之后
%
% \begin{preface}
% \lipsum[1]
% \vspace{1cm}
% \begin{flushright}
% 作者\\
% 20xx年夏于南京大学
% \end{flushright}

% \end{preface}

%%%%%%%%%%%%%%%%%%%%%%%%%%%%%%%%%%%%%%%%%%%%%%%%%%%%%%%%%%%%%%%%%%%%%%%%%%%%%%%
% 生成论文目录
\begingroup
  \let\oldaddcontentsline\addcontentsline
  \renewcommand{\addcontentsline}[3]{} % 暂时禁用目录自身的目录项
  \tableofcontents
\endgroup

%%%%%%%%%%%%%%%%%%%%%%%%%%%%%%%%%%%%%%%%%%%%%%%%%%%%%%%%%%%%%%%%%%%%%%%%%%%%%%%
% 生成插图清单。如无需插图清单则可注释掉下述语句。
\listoffigures

%%%%%%%%%%%%%%%%%%%%%%%%%%%%%%%%%%%%%%%%%%%%%%%%%%%%%%%%%%%%%%%%%%%%%%%%%%%%%%%
% 生成附表清单。如无需附表清单则可注释掉下述语句。
\listoftables

%%%%%%%%%%%%%%%%%%%%%%%%%%%%%%%%%%%%%%%%%%%%%%%%%%%%%%%%%%%%%%%%%%%%%%%%%%%%%%%
% 开始正文部分
\mainmatter

%%%%%%%%%%%%%%%%%%%%%%%%%%%%%%%%%%%%%%%%%%%%%%%%%%%%%%%%%%%%%%%%%%%%%%%%%%%%%%%
% 学位论文的正文各章建议独立成文件,统一放在根目录下的 chapter/ 目录中
% 示例:第 1 章《绪论》
% 第1章 绪论示例
\chapter{绪论}
\label{chap:intro}

\section{研究背景}
本节给出一个简单的示例背景描述,用于展示如何在独立的章节文件中撰写内容。这里可以自由撰写你的研究领域背景,例如:大模型、分布式系统或计算机视觉等。

\section{研究目的与意义}
本节说明本课题要解决的问题以及其应用价值。你可以在此处分条列出主要贡献,或介绍论文结构安排。

\section{论文结构安排}
本论文其余章节组织结构示例:
\begin{itemize}
  \item 第~\ref{chap:related} 章:相关工作;
  \item 第~\ref{chap:method} 章:方法设计;
  \item 第~\ref{chap:experiment} 章:实验与结果分析;
  \item 第~\ref{chap:conclusion} 章:总结与展望。
\end{itemize}


%%%%%%%%%%%%%%%%%%%%%%%%%%%%%%%%%%%%%%%%%%%%%%%%%%%%%%%%%%%%%%%%%%%%%%%%%%%%%%%
% 参考文献。应放在\backmatter之前。
% 推荐使用BibTeX,若不使用BibTeX时注释掉下面一句。
%\nocite{*}
\bibliography{references}


% 附录,必须放在参考文献后,backmatter前
\appendix
\chapter{附录示例}\label{app:1}
\section{main函数}
\begin{lstlisting}[language=C]
int main()
{
	return 0;
}
\end{lstlisting}


%%%%%%%%%%%%%%%%%%%%%%%%%%%%%%%%%%%%%%%%%%%%%%%%%%%%%%%%%%%%%%%%%%%%%%%%%%%%%%%
% 致谢
\begin{acknowledgement}
致谢示例

\end{acknowledgement}


%%%%%%%%%%%%%%%%%%%%%%%%%%%%%%%%%%%%%%%%%%%%%%%%%%%%%%%%%%%%%%%%%%%%%%%%%%%%%%%
% 书籍附件
\backmatter
%%%%%%%%%%%%%%%%%%%%%%%%%%%%%%%%%%%%%%%%%%%%%%%%%%%%%%%%%%%%%%%%%%%%%%%%%%%%%%%
% 作者简历与科研成果页,应放在backmatter之后
% \begin{resume}
% 论文作者身份简介,一句话即可。
% \begin{authorinfo}
% \noindent 韦小宝,男,汉族,1985年11月出生,江苏省扬州人。
% \end{authorinfo}
% 论文作者教育经历列表,按日期从近到远排列,不包括将要申请的学位。
% \begin{education}
% \item[2007年9月 --- 2010年6月] 南京大学计算机科学与技术系 \hfill 硕士
% \item[2003年9月 --- 2007年6月] 南京大学计算机科学与技术系 \hfill 本科
% \end{education}
% 论文作者在攻读学位期间所发表的文章的列表,按发表日期从近到远排列。
% \begin{publications}
% \item Xiaobao Wei, Jinnan Chen, ``Voting-on-Grid Clustering for Secure
%   Localization in Wireless Sensor Networks,'' in \textsl{Proc. IEEE International
%     Conference on Communications (ICC) 2010}, May. 2010.
% \item Xiaobao Wei, Shiba Mao, Jinnan Chen, ``Protecting Source Location Privacy
%   in Wireless Sensor Networks with Data Aggregation,'' in \textsl{Proc. 6th
%     International Conference on Ubiquitous Intelligence and Computing (UIC)
%     2009}, Oct. 2009.
% \end{publications}
% 论文作者在攻读学位期间参与的科研课题的列表,按照日期从近到远排列。
% \begin{projects}
% \item 国家自然科学基金面上项目``问题研究''
% (课题年限~2010年1月 --- 2012年12月),负责相关问题的研究。
% \end{projects}
% \end{resume}

%%%%%%%%%%%%%%%%%%%%%%%%%%%%%%%%%%%%%%%%%%%%%%%%%%%%%%%%%%%%%%%%%%%%%%%%%%%%%%%
% 生成版权及论文原创性说明
% \statement

%%%%%%%%%%%%%%%%%%%%%%%%%%%%%%%%%%%%%%%%%%%%%%%%%%%%%%%%%%%%%%%%%%%%%%%%%%%%%%%
% 生成《学位论文出版授权书》页面,应放在最后一页
% \makelicense

%%%%%%%%%%%%%%%%%%%%%%%%%%%%%%%%%%%%%%%%%%%%%%%%%%%%%%%%%%%%%%%%%%%%%%%%%%%%%%%
\end{document}
